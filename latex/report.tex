%%%%%%%% DATA LITERACY 2025 LATEX PROJECT TEMPLATE FILE %%%%%%%%%%%%%%%%%
%%% Based on the 2025 ICML template, available at https://icml.cc/Conferences/2025/AuthorInstructions %%%

\documentclass{article}

% Recommended, but optional, packages for figures and better typesetting:
\usepackage{microtype}
\usepackage{graphicx}
\usepackage{subfigure}
\usepackage{booktabs} % for professional tables

\usepackage{tikz}
% Corporate Design of the University of Tübingen
% Primary Colors
\definecolor{TUred}{RGB}{165,30,55}
\definecolor{TUgold}{RGB}{180,160,105}
\definecolor{TUdark}{RGB}{50,65,75}
\definecolor{TUgray}{RGB}{175,179,183}

% Secondary Colors
\definecolor{TUdarkblue}{RGB}{65,90,140}
\definecolor{TUblue}{RGB}{0,105,170}
\definecolor{TUlightblue}{RGB}{80,170,200}
\definecolor{TUlightgreen}{RGB}{130,185,160}
\definecolor{TUgreen}{RGB}{125,165,75}
\definecolor{TUdarkgreen}{RGB}{50,110,30}
\definecolor{TUocre}{RGB}{200,80,60}
\definecolor{TUviolet}{RGB}{175,110,150}
\definecolor{TUmauve}{RGB}{180,160,150}
\definecolor{TUbeige}{RGB}{215,180,105}
\definecolor{TUorange}{RGB}{210,150,0}
\definecolor{TUbrown}{RGB}{145,105,70}

% hyperref makes hyperlinks in the resulting PDF.
% If your build breaks (sometimes temporarily if a hyperlink spans a page)
% please comment out the following usepackage line and replace
% \usepackage{icml2023} with \usepackage[nohyperref]{icml2023} above.
\usepackage{hyperref}


% Attempt to make hyperref and algorithmic work together better:
\newcommand{\theHalgorithm}{\arabic{algorithm}}

\usepackage[accepted]{icml2025}

% For theorems and such
\usepackage{amsmath}
\usepackage{amssymb}
\usepackage{mathtools}
\usepackage{amsthm}

% if you use cleveref..
\usepackage[capitalize,noabbrev]{cleveref}

\usepackage[textsize=tiny]{todonotes}


% The \icmltitle you define below is probably too long as a header.
% Therefore, a short form for the running title is supplied here:
\icmltitlerunning{Project Report Template for Data Literacy 2025}

\begin{document}

\twocolumn[
\icmltitle{Usage Patterns of Bicycle Counting Stations in Heidelberg \\ and the Influence of External Factors}

\icmlsetsymbol{equal}{*}

\begin{icmlauthorlist}
\icmlauthor{Julian Jurcevic}{equal,first}
\icmlauthor{Martin Eichler}{equal,second}
\icmlauthor{Simon Rappenecker}{equal,third}
\icmlauthor{Tarik Eker}{equal,fourth}
\end{icmlauthorlist}

% fill in your matrikelnummer, email address, degree, for each group member
\icmlaffiliation{first}{Matrikelnummer 7167089, MSc Computer Science}
\icmlaffiliation{second}{Matrikelnummer 6009076, MSc Computer Science}
\icmlaffiliation{third}{Matrikelnummer 6324777, MSc Machine Learning}
\icmlaffiliation{fourth}{Matrikelnummer 5668988, MSc Computer Science}

% put your email addresses here. You can use initials to save space, 
% DO USE YOUR UNIVERSITY EMAIL ADDRESS!
\icmlcorrespondingauthor{JJ}{7167089julian-steffan.jurcevic@student.uni-tuebingen.de} 
\icmlcorrespondingauthor{ME}{martin.eichler@student.uni-tuebingen.de}
\icmlcorrespondingauthor{SR}{simon.rappenecker@student.uni-tuebingen.de}
\icmlcorrespondingauthor{TE}{tarik.eker@student.uni-tuebingen.de}

% You may provide any keywords that you
% find helpful for describing your paper; these ar^^e used to populate
% the "keywords" metadata in the PDF but will not be shown in the document
\icmlkeywords{Machine Learning, ICML}

\vskip 0.3in
]
% Footnote

%\printAffiliationsAndNotice{}  % leave blank if no need to mention equal contribution
\printAffiliationsAndNotice{\icmlEqualContribution} % otherwise use the standard text.

\begin{abstract}
\begin{itemize}
    \item Desribe topic
    \item Data source (just heidelberg) and external Factors
    \item sentence in one method
    \item Results briefly
\end{itemize}
\end{abstract}

\section{Introduction}\label{sec:intro}
\begin{itemize}
    \item Short introduction, but more motivational, bicycle traffic more important over the years...
    \item Name what is interesting and why it matters
    \item Shortly introduce the data, and the main method
\end{itemize}

\section{Data}\label{sec:data}
Begin describing each dataset, DO NOT REPEAT WHAT THIS SECTION IS about.
\begin{itemize}
    \item Cyclist data
    \item Describe shortly what it is and how it is physically collected, what information does it contain?
    \item Describe the fetching process and SOURCE AS HYPERLINK OR FOOTNOTE?
    \item Link to figure showing the placement in Heidelberg, evaluate the placement, describe geography, flat, etc. 
    \item Data Sanity, show a plot for a few (or only one) station, maybe also show failures and give more information about it? PLOT: SRTATION
\end{itemize}
\begin{itemize}
    \item Weather data, 
    \item Why we need this, Station data is bad, describing fetching, here we name the source
    \item OPTIONAL: Give comparision plot between temperatures and bad metrics as rain PLOT: WEATHERDATA
\end{itemize}
\begin{itemize}
    \item Accident data? Not used, but need to include it
\end{itemize}
\begin{itemize}
    \item Holiday data,
    \item We we need it, describe source, also name that there were some errors in the dataset, we have to be careful, ...
\end{itemize}
\section{Method}\label{sec:methods}
Previous works often rely on predefined rules to categorize bicycle traffic \citep{american_bike_patterns}. However, such an approach may fail to capture hybrid usage patterns.
We propose a more data-driven, implicit approach. By extracting features that quantify the 'shape' of daily, weekly, and seasonal profiles,
we map each station into a multi-dimensional feature space. This allows us to use unsupervised k-means clustering \citep{kmeans} to discover different station types. \\ \\
In a subsequent step in chapter \ref{sec:results}, we assign one of the following labels to each cluster (cf. Fig. \todo{Figure Patterns, Indices}):
\begin{itemize}
    \item \emph{Utilitarian}: Traffic showing strong commuting patterns, with two distinct peaks in traffic volume during the morning and afternoon hours.
    \item \emph{Recreational}: Traffic dominated by leisure activities. These locations exhibit a single peak in traffic volume around midday or early afternoon.
    \item \emph{Mixed}: Locations that serve a dual purpose, showing characteristics of both utilitarian and recreational usage.
\end{itemize}
% k-Means Features
To characterise the counting stations, we derive three distinct features that capture traffic patterns across different timescales: daily, weekly and seasonal
to exploit the distinct pattern visible in Figure xyz \todo{Figure pattern}.
% DPI
\subsection*{Double Peak Index (DPI)}\label{sec:dpi}
Weekday hourly profiles of some stations exhibit a double-peak structure, typically associated with morning and evening commuting.
The DPI quantifies this behaviour by identifying dominant morning (5–10 h) and evening (14–20 h) peaks and relating their magnitudes to the average midday level (8–14 h). 
Stations with clear, balanced commuting peaks yield high DPI values, whereas flat or single-peak profiles result in low scores. \\ \\ 
Formally, let $p_m$ and $p_e$ denote the magnitudes of the morning and evening peaks at hours $h_m$ and $h_e$, and let $m$ be the average midday traffic level. The DPI is defined as
\begin{align*}
    \text{DPI} &= \max\!\left( S \cdot Y \cdot D,\, 0 \right)
\end{align*}
where $S$, $Y$ and $D$ models strength, symmetry and distance, respectively:
\begin{align*}
        S &= \frac{(p_m - m) + (p_e - m)}{2} \\
    Y &= 1 - \frac{|p_m - p_e|}{\max(p_m, p_e)} \\
    D &= \min\!\left(\frac{|h_e - h_m|}{10},\, 1\right)
\end{align*}
\subsection*{Weekend Shape Difference (WSD)}\label{sec:wsd}
% WSD
Differences between weekday and weekend hourly traffic patterns provide an additional discriminator between usage types. To capture this effect, we compare the shape of the weekday and weekend hourly profiles.
Let $\mathbf{p}^{wd}$ and $\mathbf{p}^{we}$ denote the weekday and weekend hourly profiles, normalised to sum to one.
\begin{align*}
\text{WSD} =
\left\lVert
\frac{\mathbf{p}^{wd}}{\sum_h p^{wd}_h}
-
\frac{\mathbf{p}^{we}}{\sum_h p^{we}_h}
\right\rVert_2
\end{align*}
% SDI
\subsection*{Seasonal Drop Index (SDI)}\label{sec:sdi}
Finally, we consider long-term patterns. Seasonality provides a discriminator between leisure and utilitarian-oriented stations. The SDI quantifies the relative decline between high and low-usage months.
Let $I_m$ denote the monthly index values of a station. Using upper and lower quantiles to ensure robustness against outliers
\begin{align*}
q_{90} &= \text{quantile}_{0.9}(I_m) \\
q_{10} &= \text{quantile}_{0.1}(I_m)
\end{align*}
the SDI is defined as
\begin{align*}
\text{SDI} &= \frac{q_{90} - q_{10}}{q_{90}}
\end{align*}
High values indicate strong seasonal variation, whereas low values correspond to relatively stable, year-round usage.
% This is the template for a figure from the original ICML submission pack. In lecture 10 we will discuss plotting in detail.
% Refer to this lecture on how to include figures in this text.
% 
% \begin{figure}[ht]
% \vskip 0.2in
% \begin{center}
% \centerline{\includegraphics[width=\columnwidth]{icml_numpapers}}
% \caption{Historical locations and number of accepted papers for International
% Machine Learning Conferences (ICML 1993 -- ICML 2008) and International
% Workshops on Machine Learning (ML 1988 -- ML 1992). At the time this figure was
% produced, the number of accepted papers for ICML 2008 was unknown and instead
% estimated.}
% \label{icml-historical}
% \end{center}
% \vskip -0.2in
% \end{figure}

\section{Results}\label{sec:results}
\begin{itemize}
    \item describe temporal shift, describe why this is expected and why, PLOT: city,
    \item describe holidays impact, describe behaviour (similar as weekend) PLOT: show shift
    \item include also plot of different public holidays (my own one), this is just to have a funny fact (Vater Tag) PLOT: funny plot, na werden zu viele plots
    \item describe weather impact, obvious PLOT: weather
\end{itemize}

\section{Conclusion}\label{sec:conclusion}
\begin{itemize}
    \item summarize
    \item limitations
    \item problems
    \item statements that can be made
\end{itemize}

\newpage

\section*{Contribution Statement}
Add this, see original template

\bibliography{bibliography}
\bibliographystyle{icml2025}

\end{document}

% This document was modified from the files available at https://icml.cc/Conferences/2025/AuthorInstructions
% the full copyright notice is available within the file icml2025.sty